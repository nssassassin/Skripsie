\chapter*{Abstract}
\addcontentsline{toc}{chapter}{Abstract}
\makeatletter\@mkboth{}{Abstract}\makeatother

\subsubsection*{English}

Minibus taxis are a popular mode of transportation in the public sector of South Africa, but they may contribute to air pollution and pose health risks to drivers, passengers and bystanders. This thesis report aims to measure the levels of CO2, VOC, particulate matter and NOx both inside minibus taxis and in minibus taxi ranks. It also aims to identify the primary sources of air pollution in these settings and evaluate the impact of environmental factors, such as traffic congestion and weather conditions. Moreover, it investigates the potential health risks associated with exposure to air pollution in minibus taxi ranks and within minibus taxis, particularly for passengers, drivers and potential third parties. Furthermore, it aims to evaluate the effectiveness of current measures in place to reduce air pollution from minibus taxis, such as emission standards and regulations. Finally, it proposes potential strategies to mitigate the impact of from minibus taxis on public health and the environment such as implementing new technologies. This thesis report aims to provide valuable insights for policy makers, minibus taxi operators and consumers to improve air quality and protect public health.



\selectlanguage{afrikaans}

\subsubsection*{Afrikaans}

Minibustaxi’s is 'n gewilde vervoermiddel in Suid-Afrika, maar hulle kan bydra tot lugbesoedeling en gesondheidsrisiko’s vir bestuurders, passasiers en omstanders. Hierdie skripsie-verslag beoog om die vlakke van CO2, VOC, fyn stof en NOx binne minibustaxi’s en in minibustaxistaanplekke te meet. Dit streef  ook om die primêre bronne van lugbesoedeling in hierdie omgewings te identifiseer en die impak van omgewingsfaktore, soos verkeersopeenhoping en weerstoestande, te evalueer. Verder ondersoek dit die potensiële gesondheidsrisiko’s wat verband hou met blootstelling aan lugbesoedeling in minibustaxistaanplekke en binne minibustaxi’s, veral vir passasiers, bestuurders en potensiële derde partye. Verder mik dit ook om die doeltreffendheid van huidige maatreëls wat in plek is om lugbesoedeling deur minibustaxi’s te verminder, soos uitlaatgasstandaarde en regulasies, te evalueer. Laastens stel dit moontlike strategieë voor om die impak van minibustaxi’s op openbare gesondheid en die omgewing te versag, soos die implementering van nuwe tegnologieë. Hierdie skripsie-verslag mik om waardevolle insigte te verskaf vir beleidmakers, minibustaxibedrywers en verbruikers om luggehalte te verbeter en openbare gesondheid te beskerm.

\selectlanguage{english}
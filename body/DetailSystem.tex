\chapter{Detailed System Design}
\section{ESP32}

\subsection{Overview}
The ESP32-S2 is capable of multiple STA and AP modes, including simultaneous broadcasting over ESP-NOW and WiFi. This is needed to be able to have the 2 ESP boards communicate and the base station be able to send the data to a database.

\noindent
The main ESP referenced as the Base station, will be connected periodically to the internet to send the data it collects to the database of choice. 

\subsection{Software/Firmware implementation}



\subsection{UART}
The GPS unit uses UART Serial commands to send its data to the ESP32. The data received from the GPS module is in the format of NMEA strings. This is the standard format for most gps receivers.\cite{NMEA}
% Please add the following required packages to your document preamble:
% \usepackage{graphicx}
\begin{table}[!htb]
	\resizebox{\textwidth}{!}{%
		\begin{tabular}{|l|l|}
			\hline
			NMEA Sentence & Meaning \\ \hline
			GPGGA & Global positioning system fix data (time, position, fix type data) \\ \hline
			GPGLL & Geographic position, latitude, longitude \\ \hline
			GPVTG & Course and speed information relative to the ground \\ \hline
			GPRMC & Time, date, position, course and speed data \\ \hline
			GPGSA & GPS receiver operating mode, satellites used in the position solution, and DOP values. \\ \hline
			GPGSV & The number of GPS satellites in view satellite ID numbers, elevation, azimuth and SNR values. \\ \hline
			GPMSS & Signal to noise ratio, signal strength, frequency, and bit rate from a radio beacon receiver. \\ \hline
			GPTRF & Transit fix data \\ \hline
			GPSTN & Multiple data ID \\ \hline
			GPXTE & cross track error, measured \\ \hline
			GPZDA & Date and time (PPS timing message, synchronized to PPS). \\ \hline
			150 & OK to send message. \\ \hline
		\end{tabular}%
	}
	\label{tab:nmea}
	\caption{NMEA Sentences and their meanings \cite{GPSSentence}}
\end{table}
This data is sent over the uart in comma delimited messages, the uart is set to default to 9600 baud.

The ESP32-S2 has 2 hardware UARTs, one is used for debugging and communication with the device while developing and one for communicating with the GPS module. The first UART is set to 115200 baud and the second to 9600 baud. Each UART is initialized separately and the second UART is passed to the gps encoding library TinyGPS++. The first UART is only called when debugging or notices are needed. It is used to check sending of messages using ESP-NOW for example.


\subsection{i2c}
Both the Sensirion and Senseair sensors make use of the i2c bus to communicate. 

\subsection{ESP-NOW}


\subsection{SD - interface}
The SD card interface was done using SPI. The module was hard soldered to a micro SD to SD card adapter. These pins were then soldered to the ESP32's SPI and 3.3V pins.
The FS, SD and SPI libraries from Espresiff were used to interface with the already formatted micro SD card.




\section{Sensors}
\subsection{$\mathrm{CO_2}$}


% why was it chosen


\subsection{PM, $\mathrm{NO_x}$, VOC}

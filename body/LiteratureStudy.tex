\chapter{Background Study}

%Related work
%Objective
%Related Studies
%Remaining challenges

%information acquired from other sources
% Sensing, types of sensors
% How the sensors work
% Additions tech

\section{Related Work and Existing Solutions}
\subsection{Related Work}

\subsubsection{Air quality at bus stops \cite{busstop}}
This article was a case study of air quality monitoring at a bus stop in an underpass on the campus of Lancaster University. The bus stop was suspected to have high levels of air pollution due to the large number of vehicles passing through the tunnel. They used an Aeroqual AQY Micro Air Quality Station to measure the concentrations of $NO_2$, $PM_{10}$ and $PM_{2.5}$ at the bus stop. 


\subsubsection{Exposure to traffic air pollutants in taxicabs \cite{insidetaxismall}}
This study reviewed the level of pollutants present inside taxi cabs(American style taxis). The article reports that the exposure studies show that traffic related air pollutants concentrations inside taxicabs are higher than their urban background. This was a research based study.

\subsubsection{An investigation into the environmental impact of the taxi industry in Butterworth \cite{Environmentalimpact}}
This study analysed the environmental impact of the taxi industry in South Africa, by surveying a fleet of taxis in Butterworth, they do simple analogue measures such as dust gauges and soil and water analysis. 


\pagebreak
\subsection{Existing Solutions}

\subsubsection{Aeroqual AQY}
This sensor was used in a similar study done to determine the air quality at bus stops \cite{busstop}. 
The sensor solution used consists of sensors for the following along with provided ranges\cite{sensoraq}:

\begin{itemize}
	\item Particulate matter ($PM_{2.5}$ \& $PM_{10}$) 0-1000 $ \si{\micro\gram}$
	\item Ozone
	\item Nitrogen Dioxide 0-500 ppb
	\item Temperature and Relative Humidity
	\item Dew point
\end{itemize}
This sensor lacks Carbon Dioxide measuring and lacks detailed specification on ranges' error.


\subsubsection{Airthings View Plus}
This sensor suite was designed for home use, it features:
\begin{itemize}
	\item Particulate matter ($PM_{2.5}$ 0-200$ \si{\micro\gram}$
	\item Carbon dioxide 400–5000 ppm
	\item VOC
	\item Radon
\end{itemize}
This sensor was not intended to be extremely accurate and seems to be more for sensing danger than accurate measuring.
It does include a handy chart for what should be considered normal levels for the different sensors as seen in Figure~\ref{fig:airthingstable}
%\section{Air Quality Monitoring}
%For this project it is necessary to monitor the air quality and how that pertains to health conditions and what causes the rise in these parameters.
%\subsection{Choosing Monitoring Parameters}
%For monitoring it is important to choose parameters relevant to the topic at hand, in this case emissions from taxis, or for the relevance of finding data on the former, minibuses. It is found that vehicles produce the following gases\cite{gasfromvehicles}  :

%\begin{itemize}
%	\item Carbon Dioxide ($CO_2$)
%	\item Methane ($CH_4$)
%	\item Nitrous Oxides ($NO_x$)
%	\item Air Conditioning Refrigerant
%\end{itemize}

%\noindent With diesel creating 15\% more per gallon($3,78541\si{\liter}$), keeping this information in mind, the relevant gases should also be analysed for harm caused.
\noindent
%As well as these parameters, on-road activities contribute to particulate matter, particularly $PM_{2.5}$ \cite{particulatematter}


%\subsection{Effects on Human health and safe ranges}

%\subsubsection{$\mathbf{CO_2}$}
%Exposure to $CO_2$ in concentrations as low as 1000 ppm can lead to adverse effects\cite{healthrisksco2}, this is enhanced by prolonged exposure, as would likely be experienced by the driver or passenger. 


%\pagebreak

%\subsubsection{Particulate Matter }
%Short-term exposure to particulate matter in the 2.5 and 10 $ \si{\micro\gram} $ range seems to to aggravate pre-existing conditions, such as respiratory and cardiovascular conditions\cite{pmparticles}, long term exposure is irrelevant in this context.



%\begin{figure}[!htb]
%	\centering
%%	\includegraphics[width=0.7\linewidth]{body/fig/}
%	\caption{Air quality index}
%	\label{fig:tablep}
%\end{figure}


%\subsubsection{Volatile Organic Compounds}
%Short term exposure to VOCs can lead to eye and respiratory tract irritation, headaches, nausea and cancer\cite{safevocs}.
%The Environmental XPRT article on acceptable VOC levels in the air (2019) \cite{vocs} suggests that acceptable ranges for VOC would be between 300 to 500 $ %\si{\micro\gram} $ .

%\subsubsection{$\mathbf{NO_x}$}
%Symptoms from exposure to $NO_2$ include inflammation of the airways, increase susceptibility to respiratory infections and to allergens as well as aggravating pre-existing lung or heart conditions. The safe amount that should not me exceeded regularly is 200$ \si{\micro\gram} $.\cite{safenox}


%\subsection{Typical Monitoring Instrumentation}
%Studies doing similar research used pre-existing instrumentation to measure the data collected, some of which will be discussed here.





\chapter{Literature Study}

\section{Air Quality Monitoring}
For this project it is necessary to monitor the air quality and how that pertains to health conditions and what causes the rise in these prarameters.
\subsection{Choosing Monitoring Parameters}
For monitoring it is important to choose parameters relevant to the topic at hand, in this case emissions from taxis, or for the relevance of finding data on the former, minibuses. It is found that vehicles produce the following gases\cite{gasfromvehicles}  :

\begin{itemize}
	\item Carbon Dioxide ($CO_2$)
	\item Methane ($CH_4$)
	\item Nitrous Oxides ($NO_x$)
	\item Air Conditioning Refrigerant
\end{itemize}

\noindent With diesel creating 15\% more per gallon($3,78541\si{\liter}$), keeping this information in mind, the relevant gases should also be analysed for harm caused.
\noindent
As well as these parameters, on-road activities contribute to particulate matter, particularly $PM_{2.5}$ \cite{particulatematter}


\subsection{Effects on Human health and safe ranges}

\subsubsection{$\mathbf{CO_2}$}
Exposure to $CO_2$ in concentrations as low as 1000 ppm can lead to adverse effects\cite{healthrisksco2}, this is enhanced by prolonged exposure, as would likely be experienced by the driver or passenger. 


\pagebreak

\subsubsection{Particulate Matter }
Short-term exposure to particulate matter in the 2.5 and 10 $ \si{\micro\gram} $ range seems to to aggravate pre-existing conditions, such as respiratory and cardiovascular conditions\cite{pmparticles}, long term exposure is irrelevant in this context.



%\begin{figure}[!htb]
%	\centering
%%	\includegraphics[width=0.7\linewidth]{body/fig/}
%	\caption{Air quality index}
%	\label{fig:tablep}
%\end{figure}


\subsubsection{Volatile Organic Compounds}
Short term exposure to VOCs can lead to eye and respiratory tract irritation, headaches, nausea and cancer\cite{safevocs}.
The Environmental XPRT article on acceptable VOC levels in the air (2019) \cite{vocs} suggests that acceptable ranges for VOC would be between 300 to 500 $ \si{\micro\gram} $ .

\subsubsection{$\mathbf{NO_x}$}
Symptoms from exposure to $NO_2$ include inflammation of the airways, increase susceptibility to respiratory infections and to allergens as well as aggravating pre-existing lung or heart conditions. The safe amount that should not me exceeded regularly is 200$ \si{\micro\gram} $.\cite{safenox}


\subsection{Typical Monitoring Instrumentation}
Studies doing similar research used pre-existing instrumentation to measure the data collected, some of which will be discussed here.

\subsubsection{Aeroqual AQY}
This sensor was used in a similar study done to determine the air quality at bus stops \cite{busstop}. 
The sensor solution used consists of sensors for the following along with provided ranges\cite{sensoraq}:

\begin{itemize}
	\item Particulate matter ($PM_{2.5}$ \& $PM_{10}$) 0-1000 $ \si{\micro\gram}$
	\item Ozone
	\item Nitrogen Dioxide 0-500 ppb
	\item Temperature and Relative Humidity
	\item 	Dew point
\end{itemize}

\subsubsection{ScioSense APC1 Environmental Sensor} \cite{datasensor}


Most of the other studies used a combination of custom sensors, so the above will be taken as reference. 

\section{Communication}

\section{}

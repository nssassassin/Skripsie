\chapter{Literature Study}

\section{Air Quality Monitoring}
For this project it is necessary to monitor the air quality and how that pertains to health conditions and what causes the rise in these prarameters.
\subsection{Choosing Monitoring Parameters}
For monitoring it is important to choose parameters relevant to the topic at hand, in this case emissions from taxis, or for the relevance of finding data on the former, minibuses. It is found that vehicles produce the following gases\cite{gasfromvehicles}  :

\begin{itemize}
	\item Carbon Dioxide ($CO_2$)
	\item Methane ($CH_4$)
	\item Nitrous Oxides ($NO_x$)
	\item Air Conditioning Refrigerant
\end{itemize}

\noindent With diesel creating 15\% more per gallon($3,78541\si{\liter}$), keeping this information in mind, the relevant gases should also be analysed for harm caused.
\noindent
As well as these parameters, on-road activities contribute to particulate matter, particularly $PM_{2.5}$ \cite{particulatematter}


\subsection{Effects on Human health}

\subsubsection{$\mathbf{CO_2}$}


\subsubsection{Particulate Matter }

\subsubsection{Volatile Organic Compounds}

\subsubsection{$\mathbf{NO_x}$}



\subsection{Typical Monitoring Instrumentation}


\section{Communication}

\section{}

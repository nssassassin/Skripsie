\chapter{Project Planning Schedule}
\makeatletter\@mkboth{}{Appendix}\makeatother
\label{appen:appa}




\begin{table}[!htb]
	\centering
	\resizebox{0.5\textwidth}{!}{%
		
		\begin{tabular}{|l|l|}
			\hline
			13/02/2023 & Initial Skripsie Start \\ \hline
			22/02/2023 & Supervisor agreement \\ \hline
			23/02/2023 & Research \\ \hline
			03/03/2023 & Components ordered \\ \hline
			04/03/2023 & Research \\ \hline
			04/04/2023 & Some components received \\ \hline
			05/04/2023 & Begin initial coding \\ \hline
			16/04/2023 & Receive GPS, faulty \\ \hline
			30/04/2023 & Finalise design \\ \hline
			10/05/2023 & Receive 3D print \\ \hline
			15/05/2023 & Finalise coding \\ \hline
			23/05/2023 & Final Tests \\ \hline
			24/05/2023 & Reporting \\ \hline
			05/06/2023 & Hand-in \\ \hline
		\end{tabular}%
	}
	\caption{Planning schedule}
	\label{tab:plann}
\end{table}


\chapter{Outcomes Compliance}
\makeatletter\@mkboth{}{Appendix}\makeatother
\label{appen:appb}

\section{GA 1: Problem solving}
\textbf{(identify, formulate, analyse and solve complex engineering problems creatively and innovatively) }
In chapter 1, a problem was described and a solution was formulated with measurable goals and objectives. Chapter 3 shows the formulation of solutions to the parts of the problem.
\section{GA 2: Application of scientific and engineering knowledge}
\textbf{(apply knowledge of mathematics, natural sciences, engineering fundamentals and an engineering speciality to solve complex engineering problems) }
Chapter 3 shows the engineering knowledge applied to choosing the correct components, while chapter 4 shows the implementation of knowledge to achieve the outcome.
\section{GA 3: Engineering design}
\textbf{(perform creative, procedural and non-procedural design and synthesis of components, systems, engineering works, products or processes) }
In chapter 3 and 4 design steps are followed with the design and fabrication of a functioning solution.

\section{GA 4: Investigations, experiments and data analysis}
\textbf{(demonstrate competence to design and conduct investigations and experiments)  }
Chapter 3 shows the consideration for different aspects when designing and research needed to select the appropriate parts. Chapter 5 shows the testing of the final product and considerations.

\section{GA 5: Engineering methods, skills and tools, including information technology}
\textbf{(demonstrate competence to use appropriate engineering methods, skills and tools, including those based on information technology) }
This project required the knowledge to fabricate a container for the modules, to code the firmware for the various components using C++ in visual code. Data acquisition and portrayal was also necessary.

\section{GA 6: Professional and technical communication}
\textbf{(demonstrate competence to communicate effectively, both orally and in writing, with engineering audiences and the community at large)  }
This written report, written in professional form, along with an oral presentation demonstrates the
competence to effectively communicate to engineering audience and a larger community.
\section{GA 8: Individual work}
\textbf{(demonstrate competence to work effectively as an individual) }
The entirety of the work done during this project was done so individually, with only the guidance
given by the project leader, being the exception.
\section{GA 9: Independent learning ability}
\textbf{(demonstrate competence to engage in independent learning through well-developed learning skills)  }
Chapter 3 shows research ability and chapter 4 shows the implementation of external resources and API's that required furthering programming knowledge as well as working with protocols previously unknown such as the NMEA format.


\chapter{Appendix}
\makeatletter\@mkboth{}{Appendix}\makeatother

\begin{figure}[!htb]
	\centering
	\includegraphics[width=0.3\linewidth]{appendices/tablep,}
	\caption{Airthings table}
	\label{fig:airthingstableappc}
\end{figure}
